\section{Tuning}\label{sec:tuning}


\subsection{Thoughts on tuning VP}

For the simulated data, the tuning to achieve acceptable consistency seemed to mainly
involve $Q$ and $R$, 

The tuning was mainly done on the first 5000 iterations, due to constraints
in runtime.
We first tuned by only looking at the odometry results,
trying to get as good fit as possible on the first few thousand samples.
The weakness in the odometry appeared to be long turns, 
where it slowly drifted resulting in an increasing off-set.
This then appears to be what the laser measurements must handle.
Considering that the odometry appears fairly capable of following most 
maneuvers, it would seem that the noise required for the laser
updates should be fairly small in order to compensate
for the odometry drift. 

\begin{itemize}
    \item Higher $R_1$ Less landmarks
    \item Trust odometry to much causes predicted position to change making it harder to make associations
    \item Avoid detecting same object as different
    \item Make sure Q is tuned so that P contains reasonable values
    \item High Q causes initial pose offset that stays uncorrected
    \item To many landmarks indicate that our position has drifted such that the same landmark is detected again
    \item To few landmarks indicate that we treat different landmarks as the same, either indicates drift or to high noise
    \item To high $\sigma_3$ introduces pose error, to low makes the system react to slowly to turns
    \item Hard to get NIS above lower limit while matching GNSS trajectory
    \item Smaller JCBB alphas increase variance in NIS and decrease number of landmarks
    \item Large R, overfit and NIS becomes small, few landmarks
\end{itemize}



\begin{figure}
    \centering
    \includegraphics[clip, trim= 0cm 0cm 0cm 0cm, width = \textwidth]{figures/sim_NIS_NEES.eps}
    \caption{Consistency for simulated dataset}
	\label{fig:1_1}
\end{figure}
\begin{figure}
    \centering
    \includegraphics[clip, trim= 0cm 0cm 0cm 0cm, width = \textwidth]{figures/sim_results.eps}
    \caption{Result and RMSE for simulated dataset}
	\label{fig:1_2}
\end{figure}


\section{Consistency}

\subsection{Fit to GNSS}


To try to identify any significant offset between our estimates
and the GNSS measurements we transformed the GNSS measurements onto our estimates.
The optimal rotation was found using SVD,

\begin{align*}
    H = 
    \begin{bmatrix}
        \mathbf{Lo_{gnss}} & \mathbf{La_{gnss}}
    \end{bmatrix}
    \begin{bmatrix}
        \mathbf{Lo_{est}} \\ \mathbf{La_{est}}
    \end{bmatrix}
    , \quad
    U, S, V = SVD(H)
    , \quad
    R = VU^T
\end{align*}

whereas the translation was 
solved by OLS.

\begin{align*}
    T = \frac{1}{N} \sum_{1}^{N} 
    \begin{bmatrix}
        Lo_{gnss_i} \\ La_{gnss_i}
    \end{bmatrix}
    - \mathbf{R}
    \frac{1}{N} \sum_{1}^{N} 
    \begin{bmatrix}
        Lo_{est_i} \\ La_{est_i}
    \end{bmatrix}
\end{align*}

We observed some transformation when iterating over less data
(5-10k iterations). When iterating over more data (20k+ iterations),
the optimal transformation became close to negligible, as seen in figure (blabla).
Our understanding from this is that any drift is incurred mainly
from turning. With proper tuning, the incurred drift is seemingly negated
at a later time when turning in the opposite direction or when we are able
to correct our trajectory by making correct joins from laser measurements.
In other words,
when driving around at random, we get a good track. If however this 
was a NASCAR dataset, we would probably have to tune differently to 
avoid an increasing offset.

To compare with GNSS we computed NIS by using the odometry predictions
as if we would update using GNSS. We paired a prediction with a GNSS
measurement by locating the closest GNSS sample to each odometry measurement
that was also within 0.01 seconds (chosen by a little trial and error) 
of eachother.